\setlength{\parskip}{0pt}

\clearpage
\chapter*{Abstract}

\textbf{EN}\\

This paper examines consensus in distributed systems and the mechanisms used to achieve it, with a focus on the Raft consensus algorithm. First, the theoretical basis required to understand consensus mechanisms is laid out, introducing the reader to aspects of distributed system design that tailor to strong consistency. Then, an overview of the Raft consensus algorithm is given, detailing its design and showing its utility in modern distributed systems. Finally, a novel implementation of a simple distributed key-value store using Raft is presented. The paper examines the store's architecture as well as the tools used to develop it and guides the reader through a series of experiments, proving that it can achieve strong consistency with appealing levels of fault-tolerance. As a result, this paper functions as a comprehensive introduction to the Raft consensus algorithm and provides an implementation that proves Raft's claims and can serve as a basis for building distributed databases.\\

\textbf{EL}\\

Αυτή η εργασία εξετάζει το Consensus στα κατανεμημένα συστήματα και τους μηχανισμούς που χρησιμοποιούνται για την επιτευξή του, εστιάζοντας στον αλγόριθμο Raft. Αρχικά, παρου-σιάζεται η θεωρητική βάση που απαιτείται για την κατανόηση των μηχανισμών consensus, εισάγοντας τον αναγνώστη σε πτυχές του σχεδιασμού κατανεμημένων συστημάτων που δίνουν έμφαση στην ισχυρή συνέπεια. Στη συνέχεια, δίνεται μια επισκόπηση του αλγορίθμου Raft, περιγράφοντας λεπτομερώς τον σχεδιασμό του και δείχνοντας τη χρησιμότητά του σε σύγχρονα κατανεμημένα συστήματα. Τέλος, παρουσιάζεται η καινοτόμα υλοποίηση μίας απλής κατανε-μημένης βάσης δεδομένων χρησιμοποιώντας το Raft. Η εργασία εξετάζει την αρχιτεκτονική της βάσης καθώς και τα εργαλεία που χρησιμοποιήθηκαν για την ανάπτυξή της και καθοδηγεί τον αναγνώστη μέσα από μια σειρά πειραμάτων, αποδεικνύοντας ότι μπορεί να επιτύχει ισχυρή συνέπεια με ελκυστική ανοχή σφαλμάτων. Ως αποτέλεσμα, αυτή η εργασία λειτουργεί ως μια πλήρης εισαγωγή στον αλγόριθμο Raft και παρέχει μια υλοποίηση που αποδεικνύει τους ισχυρισμούς του και μπορεί να χρησιμοποιηθεί ως βάση για τη δημιουργία κατανεμημένων βάσεων δεδομένων.